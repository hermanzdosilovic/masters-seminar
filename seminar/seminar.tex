\documentclass[times, utf8, seminar]{fer}
\usepackage{booktabs}
\usepackage{pdfpages}
\usepackage{listings}
\usepackage{mathtools}
\usepackage{algorithmic}
\usepackage{algorithm}
\usepackage{amsmath}
\usepackage{relsize}
\usepackage[colorinlistoftodos,prependcaption,textsize=tiny]{todonotes}

\renewcommand{\lstlistingname}{Isječak}
\DeclareMathOperator*{\argmax}{arg\,max}

\begin{document}

\lstset{
    basicstyle=\linespread{1.2}\ttfamily\footnotesize,
    keepspaces=true,
    numbers=left,
    frame=single,
    showspaces=false,
    numberstyle=\ttfamily,
    columns=flexible,
    extendedchars=true,
    inputencoding=utf8,
    literate={®}{{\textregistered}}1,
}

\voditelj{prof. dr. sc. Domagoj Jakobović}

\title{
    Sustavi za upravljanje heterogenom flotom ljudi i robota u logističkim centrima
}

\author{Herman Zvonimir Došilović}

\maketitle

\tableofcontents

\chapter{Uvod}
Logistički centri su dinamički i stohastički logistički sustavi
čija je temeljna zadaća prostornovremenska transformacija
dobara koja se odvija u procesima: (1) transporta, pregrupiranja i skladištenja gdje
su bitni procesi tokova dobara. (2) Pakiranja i signiranja gdje su bitni procesi
pomaganja tokovima dobara. (3) Dostavljanja i obrada naloga gdje su bitni procesi
tijekova informacija. \citep{Paladin, buntak2012medjusobni}

Razvojem informacijskih tehnologija elektroničke trgovine \engl{ecommerce}
 posljednjih su godina
doživjele veliki zamah u razvoju i poslovanju. Međutim, sa sve većim brojem narudžbi,
logistički procesi elektroničkih trgovina postaju usko grlo kada govorimo o 
ispunjenju narudžbi i njihovih rokova. Također, logistički procesi mogu uzrokovati
probleme poput sporih, neispravnih, izgubljenih, oštećenih i pogrešnih isporuka. 
Pored toga cijena logističkog procesa može iznositi
i do 40\% cijene korisnikove narudžbe.
Automatizacija logističkih procesa nudi poboljšanja učinkovitosti, smanjenje troškova
i smanjenje vremena posluživanja, međutim, velika raznolikost i promjenljivost
narudžbi čini implementaciju automatizacije teškom. U obzir treba uzeti
ne samo tip i količinu neke naručene robe, nego i njihovu veličinu, njihov 
oblik, masu, pa čak i svojstva i posebnosti pakiranja. Također, jednom
uspostavljeni automatizirani proces mora se moći lako prilagoditi 
raznim promjenama u logističkom centru. To znači da se automatizirani sustavi
moraju lako skalirati i da moraju biti fleksibilni u svakom trenutku.
Kada govorimo o automatizaciji logističkih procesa elektroničkih trgovina, onda
razlikujemo automatizaciju protoka informacija i automatizaciju protoka robe.
Robotika nudi rješenje u automatiziranju protoka robe i roboti danas uspješno
ostvaruju zadatke koji se ponavljaju na fiksnim pozicijama sa željenom brzinom,
preciznosti i tereta. Robotika, također, nudi rješenje koje će
poboljšati učinkovitost, skalabilnost i fleksibilnost logističkih procesa.
Mobilni roboti omogućuju automatizirano izvršavanje zadataka na pozicijama
koje nisu više fiksne, međutim, tada je potrebno uskladiti suradnju između robota
i ljudi. \citep{huang2015robotics}

\pagebreak

Ovaj seminarski rad predstavit će i definirat će jedan stvarni problem koji
se pojavljuje uvođenjem mobilnih i autonomnih robota u logističke centre kako bi se
optimirala jedna faza logističkog procesa koja uključuje prikupljanje
artikala narudžbe. Osim toga, napravit će pregled područja i pregled dosadašnjih
metoda koje rješavaju sličan problem.

\chapter{Opis problema}
\section{Ulazni podaci}
\section{Očekivani izlazni podaci}

\chapter{Raspoređivanje}

\chapter{Zaključak}

\bibliography{literatura}
\bibliographystyle{fer}

\begin{sazetak}

\kljucnerijeci{upravljanje flotom,raspoređivanje}
\end{sazetak}

\engtitle{Management systems for heterogeneous fleet of humans and robots in logistic centers}
\begin{abstract}

\keywords{fleet management,scheduling}
\end{abstract}

\end{document}
