\documentclass[times, utf8, seminar]{fer}
\usepackage{booktabs}
\usepackage{pdfpages}
\usepackage{listings}
\usepackage{mathtools}
\usepackage{algorithmic}
\usepackage{algorithm}
\usepackage{amsmath}
\usepackage{relsize}
\usepackage[colorinlistoftodos,prependcaption,textsize=tiny]{todonotes}

\renewcommand{\lstlistingname}{Isječak}
\DeclareMathOperator*{\argmax}{arg\,max}

\begin{document}

\lstset{
    basicstyle=\linespread{1.2}\ttfamily\footnotesize,
    keepspaces=true,
    numbers=left,
    frame=single,
    showspaces=false,
    numberstyle=\ttfamily,
    columns=flexible,
    extendedchars=true,
    inputencoding=utf8,
    literate={®}{{\textregistered}}1,
}

\voditelj{prof. dr. sc. Domagoj Jakobović}

\title{
    Sustavi za upravljanje heterogenom flotom ljudi i robota u logističkim centrima
}

\author{Herman Zvonimir Došilović}

\maketitle

\tableofcontents

\chapter{Uvod}
Logistički centri su dinamički i stohastički logistički sustavi
čija je temeljna zadaća prostornovremenska transformacija
dobara koja se odvija u procesima skladištenja, pakiranja, dostavljanja i drugih.
\citep{Paladin, buntak2012medjusobni}

Logistički procesi internet trgovina \engl{ecommerce} najčešće se odvijaju
u tri faze. Prva faza je transport dobara od proizvođača do
logističkih centara. Druga faza uključuje zaprimanje narudžbi, prikupljanje i pakiranje naručenih artikala
u logističkim centrima. Treća faza uključuje prijevoz naručene robe od logističkog centra
do naručitelja. U današnjem svijetu sve je više
raznovrsnijih narudžbi stoga opisana druga faza predstavlja
usko grlo logističkih procesa internet trgovina.
Prikupljanje naručene robe, odnosno artikala, je naporan rad
koji uključuje kretanje, operatera i robe, na velike udaljenosti.
Robotika nudi rješenje koje će poboljšati učinkovitost, skalabilnost
i fleksibilnost logističkih procesa. \citep{huang2015robotics}

Ovaj seminarski rad predstavit će stvarni problem raspoređivanja zadataka
robotima i ljudima u drugoj fazi logističkih procesa, i napravit će
pregled dosadašnjih metoda koje rješavaju sličan problem u području
raspoređivanja.

\chapter{Zaključak}

\bibliography{literatura}
\bibliographystyle{fer}

\begin{sazetak}

\kljucnerijeci{upravljanje flotom,raspoređivanje}
\end{sazetak}

\engtitle{Management systems for heterogeneous fleet of humans and robots in logistic centers}
\begin{abstract}

\keywords{fleet management,scheduling}
\end{abstract}

\end{document}
